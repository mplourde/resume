\documentclass{tccv}
\usepackage[english]{babel}

\begin{document}
%\topskip0pt
%\vspace*{\fill}

\part{Matthew Plourde}

\section{Work experience}

\begin{eventlist}
\item{April 2011 -- Present}
     {Sci-Tek Consultants, Philadelphia Office of Watersheds}
     {Analyst \& Software Developer }

During my time at the Office of Watersheds I've serviced the data management and analysis needs of water resource engineers across several specialized teams. My projects have included
\end{eventlist}

\begin{renumerate}
\item a web-based quality assurance and analytics platform used by engineers to manage, evaluate, and draw insight from stormwater infrastructure monitors and metrics (R, Shiny Server, PostgreSQL, Amazon Web Services)
\item a fully concurrent version control system for hydrodynamic models, which includes tools for generating, comparing, and merging such models through a graphical interface (Python, SQLServer)
\item text parsers to validate and summarize complex, semi-structured data
\item a statistical calibration application for hydrodynamic models (R)
\item signal processing tools for decomposing, analyzing, and extrapolating tidal signals in the Delaware River (Octave)
\item a three dimensional dispersion rate estimator for river pollutants (R)
\item several report generation tools (R, Sweave, knitr)}
%\item the backend for a public facing, real-time river quality reporting app (\href{http://www.phillywatersheds.org/what_were_doing/documents_and_data/live_data/csocast}{CSOCast}) (Python, SQLite)
\item numerous ad-hoc databases, post-processors, and data visualization tools (PostgreSQL, Python, R)

\end{renumerate}

\section{Education}

\begin{yearlist}

\item{2006 -- 2011}
     {BS Mathematics}
     {Drexel University, Philadelphia}

\item{2004 -- 2006}
     {Linguistics, Philosophy}
     {Temple University, Philadelphia}

\end{yearlist}





\section{Publications}

\begin{yearlist2}

\item{2012}
     {Geoprocessing Tools for Surface and Basement Flooding Analysis in SWMM}
     {White, Knighton, Martens, Plourde, \mbox{Rajan}; Pragmatic Modeling of Urban Water Systems, Monograph 21}


\end{yearlist2}


\personal
    {311 N Randolph Street, Apt 4 \newline Philadelphia, PA 19106}
    {(484) 332 0087}
    {plourde.m@gmail.com}
    

\section{Courses and Exams}

\begin{yearlist2}

\item{2014}
     {Machine Learning, Coursera}
     {Neural Nets, SVMs, PCA, K-Means}

\item{2011}
     {Actuary Exam P/1}
     {Applied probability}


\end{yearlist2}


\section{Primary Technologies}

\begin{factlist}

\item{Experienced}
     {R, Python}

\item{Effective}
     {SQL (SQLServer, PostgreSQL), Regular Expressions, Vim}
     
\item{Comfortable}
     {Ubuntu Server, git, \LaTeX, Octave}
     
\item{Learning}
     {Javascript, HTML5/CSS, Amazon Web Services, Apache Spark}

\end{factlist}

\section{Software Library Experience}

\begin{factlist}

\item{R}
     {ggplot2, Shiny, data.table, dplyr, RODBC, testthat, devtools, rgdal, sp, knitr, grid, gridSVG, SparkR, caret}

\item{Python}
     {itertools, SQLAlchemy, arcpy, scipy, sqlite3, unittest, psycopg2, wxPython, matplotlib}

\item{Octave}
     {signal}

\end{factlist}


%\section{Developer Credo}
%I \textsc{believe \dots}
%\begin{itemize}[itemsep=2pt,parsep=2pt]
%\item thorough unit testing is the key to peace of mind and %confident updates
%\item DRY code is both easier to maintain \emph{and} easier to read
%\item the number of lines allotted to a task should be roughly %proportional to its complexity (or to say it another way, deep nesting without whitespace is sociopathic)
%\item eloquence improves maintainability and job satisfaction,
%\item but there is a difference between eloquent and hackish, and eloquent is not strictly a function of brevity
%\item software systems or analyses
%should be fully reproducible and this constitutes the first step of any documentation process
%\end{itemize}



\section{Hobbies \& Interests}
\begin{renumerate}
\item Classical guitar
\item Providing support for the R language on \href{http://stackoverflow.com/users/433829/matthew-plourde}{stackoverflow.com}}
\end{renumerate}
%\vspace*{\fill}

\end{document}

